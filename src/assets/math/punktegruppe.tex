%%%%%%%%%%%%%%%%%%%%%%%%%%%%%%%%%%%%%%%%%%%%%%%%%%%%%%%%%%%%%%%%%%%%%%%%%%%%%%%
%			Dateikopf
%%%%%%%%%%%%%%%%%%%%%%%%%%%%%%%%%%%%%%%%%%%%%%%%%%%%%%%%%%%%%%%%%%%%%%%%%%%%%%%

\documentclass[10pt,a4paper]{article}
\usepackage{amssymb,amsmath, amsthm}													% AMS Kram
\usepackage{german}																						% Man spricht Deutsch
\usepackage[latin1]{inputenc}

\usepackage{latexsym,enumerate,fancyhdr,textcomp,makeidx} 

%\usepackage{showidx} %prints index entries on margin
                      % - useful for index proof-reading

%%%%%		PostScript Tricks

\usepackage{pstricks, pst-plot}


%%%%%		Seitengr��e

%\textwidth 16cm
%\textheight 21cm
%\topmargin 0cm
%\oddsidemargin 0cm
%\evensidemargin 0cm


%%%%%		Kopf- und Fu�zeile

\pagestyle{fancy}
\lhead{}
%\lhead{\leftmark}
\chead{}
\rhead{}
%\rhead{\rightmark}
\lfoot{}
\cfoot{\thepage}
\rfoot{}


%%%%%		?????????

%\renewcommand{\headrulewidth}{0.4pt}
%\renewcommand{\footrulewidth}{0.4pt}
%\renewcommand{\baselinestretch}{1.05}
%\renewcommand{\arraystretch}{.9}
%\renewcommand{\rmdefault}{\sfdefault}
%\renewcommand{\textrm}{\textsf}
\parindent = 0pt 								% Keine Paragraphen


%%%%%		Theoremumgebungen

\newtheorem{theorem}{Theorem}%[section]
\newtheorem{definition}[theorem]{Definition}
\newtheorem{lemma}[theorem]{Lemma}
\newtheorem{example}[theorem]{Beispiel}
\newtheorem{remark}[theorem]{Anmerkung}
\newtheorem{corollary}[theorem]{Korollar}


\newtheorem*{theorem0}{Theorem}
\newtheorem*{definition0}{Definition}
\newtheorem*{lemma0}{Lemma}
\newtheorem*{example0}{Beispiel}
\newtheorem*{remark0}{Anmerkung}
\newtheorem*{corollary0}{Korollar}


%%%%%		Kurzkommandos

\newcommand{\N}{\mathbb N}
\newcommand{\Z}{\mathbb Z} 
\newcommand{\R}{\mathbb R} 
\newcommand{\Q}{\mathbb Q} 
\newcommand{\C}{\mathbb C} 
\newcommand{\K}{\mathbb K} 
\newcommand{\A}{\cal A}
\newcommand{\E}{\mathcal E}
\newcommand{\F}{\cal F}
\newcommand{\G}{\cal G}
\newcommand{\cO}{\mathcal O}
\newcommand{\e}{\epsilon}

%%%%%		Trennungsregeln

\hyphenation{ho-meo-mor-phic}




%%%%%		Spezifikation

\makeindex
\title{Die Punktegruppe-Operation}
\author{Richard Lindner}
\date{99.09.9999}

%%%%%%%%%%%%%%%%%%%%%%%%%%%%%%%%%%%%%%%%%%%%%%%%%%%%%%%%%%%%%%%%%%%%%%%%%%%%%%%
%			Beginn des eigentlichen Textes
%%%%%%%%%%%%%%%%%%%%%%%%%%%%%%%%%%%%%%%%%%%%%%%%%%%%%%%%%%%%%%%%%%%%%%%%%%%%%%%

\begin{document}


%%%%%		Titelseite

% \thispagestyle{empty}
% \maketitle 		% Gibt Titel, Name und Datum an.
% \begin{abstract} \end{abstract}
% \newpage


%%%%%		Inhaltsverzeichnis

% \tableofcontents
% \newpage


%%%%%%%%%%%%%%%%%%%%%%%%%%%%%%%%%%%%%%%%%%%%%%%%%%%%%%%%%%%%%%%%%%%%%%%%%%%%%%%
%%%%%%%%%%%%%%%%%%%%%%%%%%%%%%%%%%%%%%%%%%%%%%%%%%%%%%%%%%%%%%%%%%%%%%%%%%%%%%%
%%%%%%%%%%%%%%%%%%%%%%%%%%%%%%%%%%%%%%%%%%%%%%%%%%%%%%%%%%%%%%%%%%%%%%%%%%%%%%%

\lhead{Vortrag 2: Die Punktegruppe-Operation}
\rhead{Richard Lindner}

\section{Die Punktegruppe}
\subsection{Elliptische Kurven}
Eine \textit{elliptische Kurve} $\E$ �ber einem K�rper $\K$ kann, wenn der K�rper nicht die Charakteristik $2$ hat, mit der Weierstrass Normalform
$$
\E : y^2= x^3 + ax^2 + bx + c
$$
beschrieben werden. Hierbei sind $a,b,c \in \K$ und das rechte Polynom $3$. Grades in $x$ hat keine doppelten Nullstellen. 

\begin{example}2 Elliptische Kurven �ber $\R$ \\

\begin{tabular}{cc}

\psset{unit=1cm} 
\begin{pspicture}(-3,-2.5)(3,2.5)
  \psaxes{->}(0,0)(-2,-2.5)(2,2.5)
  \psplot{-1}{0}{x x x mul mul x sub sqrt} 					% Linke Schlaufe
  \psplot{-1}{0}{x x x mul mul x sub sqrt -1 mul} 

  \psplot{1}{2}{x x x mul mul x sub sqrt} 					% Rechte Schlaufe
  \psplot{1}{2}{x x x mul mul x sub sqrt -1 mul} 
\end{pspicture}
& 
\begin{pspicture}(-3,-2.5)(3,2.5)
  \psaxes{->}(0,0)(-2,-2.5)(2,2.5)
  \psplot{-1.3245}{2}{x x x mul mul x sub 1 add sqrt} 
  \psplot{-1.3245}{2}{x x x mul mul x sub 1 add sqrt -1 mul} 
\end{pspicture} \\
\\
$\E_1 : y^2 = x^3-x$  & $\E_2 : y^2 = x^3-x+1$ \\
\end{tabular}
\end{example}

Die Menge einer solchen Punktgruppe stellen wir uns wie folgt vor.

\begin{definition}\textbf{Menge der Punktegruppe} \ $\left( \E (\K), + \right)$\\
Die Punktegruppe einer elliptischen Kurve $\E$ �ber $\K$ ist die Menge der Punkte, die auf der Kurve liegen und der Fernpunkt $\cO := (\infty,\infty)$.
$$
\E(\K) := \left\{ P=(x,y) \in \K^2 : y^2=x^3+ax^2+bx+c  \right\} \cup \left\{ \cO \right\}
$$

\end{definition}

Die Operation auf dieser Menge, die wir mit $+$ bezeichnen, wollen wir uns zun�chst graphisch vorstellen. Die Summe zweier Punkt $P$ und $Q$ wird dabei konstruiert, indem man eine Gerade $\overline{PQ}$ durch $P$ und $Q$ legt und dann den $3.$ Schnittpunkt mit $\E$ an der $X$-Achse spiegelt. Dabei gibt es folgende Ausnahmen zu beachten: \\

--- Addiert man einen Punkt $P$ zu sich selbst ist die Gerade $\overline{PP}$ nicht eindeutig bestimmt. Damit die Addition stetig bleibt nimmt man die Tangente an $\E$ in $P$.\smallskip

--- Sind $P$ und $Q$ senkrecht �bereinander, so hat die Gerade $\overline{PQ}$ und $\E$ keinen dritten Schnittpunkt. Sie schneiden sich im Fernpunkt $\cO$.\smallskip

--- Addiert man zu einem Punkt $P$ den Fernpunkt $\cO$, dann stellt man sich $\cO$ �ber $P$ vor. Die Gerade $\overline{P \cO}$ schneidet also $\E$ genau gegen�ber von $P$ und nach dem Spiegeln ergibt die Addition $P + \cO = P$. \smallskip


\newpage

Was wir uns nun �berlegen wollen ist, da� $\left( \E (\K), + \right)$ mit der geometrisch beschriebenen Addition wirklich eine \textit{Gruppe} ist. Diese Gruppe ist sogar \textit{kommutativ}.\\

\begin{itemize}
\item[(i)] \textit{Abgeschlossenheit \ $+: \E (\K)^2 \rightarrow \E (\K)$}\smallskip

Da wir mit unserer Addition nie die Menge $\E (\K)$ verlassen gibt es hier kein Problem.

\item[(ii)] \textit{Neutrales Element} \ $P + 0 = P$ \smallskip

Schon w�hrend der Definition fiel uns auf, da� der Fernpunkt beliebig oft auf einen Punkt addiert werden kann ohne diesen zu �ndern. $\cO$ ist das neutrale Element.

\item[(iii)] \textit{Inverse Elemente} \ $ -P$ \smallskip

Das inverse Element zu jedem Punkt ist sein Spiegelbild an der $X$-Achse. Da eine Gerade durch zwei �bereinanderliegende Punkte zum Fernpunkt, dem neutralen Element l�uft.

\item[(iv)] \textit{Kommutativit�t} \ $P + Q = Q + P$ \smallskip

Dies gilt weil wir beim Addieren immer als erstes eine Gerade durch Beide Punkte ziehen und die Gerade gleichbleibt, egal mit welchem der beiden Punkte man anf�ngt.

\item[(v)] \textit{Assoziativit�t} \ $P + (Q + R) = (P + Q) + R$ \smallskip

Die Assoziativit�t werden wir uns an einem Bild verdeutlichen, aber nicht formal beweisen, da dies sehr technisch ist.


\begin{center}

\psset{unit=3.5cm} 
\begin{pspicture}(-1.3,-1.3)(1.3,1.3)
  \psset{linewidth=0.5pt}
  \psaxes[Dy=1,Dx=2]{->}(0,0)(-1.3,-1.3)(1.3,1.3)
  \psplot[linewidth=1pt]{-1}{0}{x x x mul mul x sub sqrt} 		% Linke Schlaufe
  \psplot[linewidth=1pt]{-1}{0}{x x x mul mul x sub sqrt -1 mul} 

  \psplot[linewidth=1pt]{1}{1.3}{x x x mul mul x sub sqrt} 					% Rechte Schlaufe
  \psplot[linewidth=1pt]{1}{1.3}{x x x mul mul x sub sqrt -1 mul} 

  \rput[lb](-1.04,0.28){$P$}
  \rput[lb](-0.21,0.48){$Q$}
  \rput[lb](-0.65,0.64){$R$}
  \rput[lb](1.24,0.71){$-(P+Q)$}
  \rput[lb](1.26,-0.81){$P+Q$}
  \rput[lb](1.03,0.03){$Q+R=-(Q+R)$}
  \rput[lb](-0.02,0.13){$-(P+Q+R)$}

  \rput[lb](-1.14,-0.1){$-1$}
  \rput[lb](0.94,-0.1){$1$}

  \psplot{-1.3}{1.3}{x .25 mul .5 add} 							% Linie PQ
  \psplot[linestyle=dashed]{-1.3}{1.3}{x -0.370060187 mul 0.370060187 add} 		% Linie -(Q+R)R
  \psplot{-1.3}{1.3}{x -0.751812641 mul 0.120586693 add} 		% Linie -(P+Q+R)
  \psplot[linestyle=dashed]{-1.3}{1.3}{x -0.131748658 mul 0.131748658 add} 		% Linie -(P+Q+R)


\end{pspicture}

\end{center}

\end{itemize}

\newpage

\subsection{Additionsformeln f�r Charakteristik $> 3$}
Nun werden wir unsere Addition noch genauer �ber \textit{Additionformeln} beschreiben. Die wohl einfachste Formel besch�ftigt sich mit dem Invertieren.\\

Die \textbf{Inversen} Elemente in unserer Gruppe findet man mit einer Spiegelung an der $X$-Achse, sprich man dreht das Vorzeichen der $Y$-Koordinate um. 
$$ -P = -(x_P,y_P) = (x_P,-y_P) $$

Wir wollen uns f�r diesen Abschnitt auf Elliptische Kurven �ber einem K�rper $\K$ mit Charakteristik gr��er $3$ beschr�nken, da sich diese in der kurzen Weierstrassform schreiben lassen und damit die Formeln angenehmer werden:
$$ \E : y^2 = x^3 + ax + b $$

Das Addieren \textbf{verschiedener} Punkte kommt zuerst. Seien $P = (x_P,y_P), Q = (x_Q,y_Q), R = (x_R,y_R)$ im Verh�ltnis
$$
P \neq Q \hspace{3cm} P+Q = R
$$
dann berechnet sich $R$ wie folgt:\\

Ist $x_P = x_Q$, sprich die Punkte sind �bereinander, dann sind $P$ und $Q$ invers zueinander $R = \cO$.\\

Sonst ist $x_P \neq x_Q$ und wir k�nnen die Steigung der Gerade $\overline{PQ}$ berechnen:
$$
s = \frac{y_Q-y_P}{x_Q-x_P}
$$

$R$ berechnet sich dann aus dieser Steigung $s$:

$$ x_R = s^2 - x_P - x_Q $$
$$ y_R = s \cdot (x_P - x_R) - y_P $$

\vspace{1cm}
Nun wollen wir 2 \textbf{gleiche} Punkte addieren, also einen Punkt verdoppeln. Sei dazu $P = (x_P,y_P), R = (x_R,y_R)$ im Verh�ltnis
$$
2\cdot P = R
$$

Die Steigung der Tangente an $\E$ in $P$ ergibt sich wie folgt:
$$
s = \frac{3x_P^2+a}{2y_P}
$$
ist $y_P$ hier 0, dann sind steht die Tangente senkrecht und schneidet $\E$ nicht mehr $R=\cO$. Sonst kann man aus der Steigung genau wie zuvor auch $R$ berechnen:
$$ x_R = s^2 - 2x_P $$
$$ y_R = s \cdot (x_P - x_R) - y_P $$

\subsection{Additionsformeln f�r Charakteristik $=2$}

Zuletzt �berlegen wir uns noch, da� das Finden des Punktes $P+Q$ letztlich dem L�sen eines nichtlinearen Gleichungsystems entspricht. Man sucht n�mlich den Punkte, der die (lineare) Geradengleichung, beschrieben durch $P$ und $Q$, erf�llt und auch die (nichtlineare) Gleichung, die alle Punkte auf der Elliptische Kurve charakterisiert.\\

Die Berechnung von \textbf{Inversen} ist in dieser Gruppe etwas anders als vorher: 
$$ \hspace{1cm} -P = (x_P,y_P+x_P) $$

F�r K�rper der Charakteristik $2$ mu� man den Begriff von "`Gerade durch $P$ und $Q$"' in diesem Kontext erweitern und daraus ergeben sich dann die nun folgenden Formeln. Elliptische Kurven �ber diesen K�rpern lassen sich auch nicht in die zum Rechnen angenehme kurze Weierstrassform bringen, sondern sie werden beschrieben mit:
$$ \E : y^2 + xy = x^3 + ax^2 + b $$

Das Addieren \textbf{verschiedener} Punkte kommt erneut zuerst. Seien $P = (x_P,y_P), Q = (x_Q,y_Q), R = (x_R,y_R)$ im Verh�ltnis
$$
P \neq Q \hspace{3cm} P+Q = R
$$
dann berechnet sich $R$ wie folgt:\\

$$ s := \frac{y_P+y_Q}{x_P+x_Q} $$
$$ x_R = s^2 + s + x_P + x_Q + a $$
$$ y_R = s \cdot (x_P + x_R) +x_R + y_P $$

\vspace{1cm}

F�r das Addieren von zwei \textbf{gleichen} Punkten, also die Verdoppelung von $P = (x_P,y_P)$
$$
2\cdot P = R
$$
ergibt sich:

$$ s = \frac{x_P+y_p}{x_P} $$
$$ x_R = s^2 + s + a = x_P^2 + \frac{b}{x_P^2}$$
$$ y_R = x_P^2 + s \cdot x_R +x_R$$




\end{document}


