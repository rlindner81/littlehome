\documentclass{beamer}

\usepackage{pstricks, pst-plot}


\newcommand{\N}{\mathbb N}
\newcommand{\Z}{\mathbb Z} 
\newcommand{\R}{\mathbb R} 
\newcommand{\Q}{\mathbb Q} 
\newcommand{\C}{\mathbb C} 
\newcommand{\K}{\mathbb K} 
\newcommand{\A}{\cal A}
\newcommand{\E}{\mathcal E}
\newcommand{\F}{\cal F}
\newcommand{\G}{\cal G}
\newcommand{\cO}{\mathcal O}
\newcommand{\e}{\epsilon}

\parindent = 0pt 								% Keine Paragraphen


\mode<presentation>
{
  %\usetheme{Warsaw}
  \usetheme{Darmstadt}
  % oder ...
  
  \setbeamercovered{transparent}
  % oder auch nicht
}


\usepackage[german]{babel}
% oder was auch immer

\usepackage[latin1]{inputenc}
% oder was auch immer

\usepackage{times}
\usepackage[T1]{fontenc}
% Oder was auch immer. Zu beachten ist, das Font und Encoding passen
% m�ssen. Falls T1 nicht funktioniert, kann man versuchen, die Zeile
% mit fontenc zu l�schen.


\title[Punktegruppe] % (optional, nur bei langen Titeln n�tig)
{Die Punktegruppe-Operation}

\subtitle
{Elliptische Kurven Kryptographie Seminar} % (optional)

\author{Richard ~Lindner}
% - Der \inst{?} Befehl sollte nur verwendet werden, wenn die Autoren
%   unterschiedlichen Instituten angeh�ren.

\institute
{Institut f�r Theoretische Informatik\\ 
Technische Universit�t Darmstadt}

\date[Seminarvortrag] % (optional)
{28.04.2005 / Seminarvortrag Nr. $2$}


\subject{Informatik}
% Dies wird lediglich in den PDF Informationskatalog einf�gt. Kann gut
% weggelassen werden.


% Falls eine Logodatei namens "university-logo-filename.xxx" vorhanden
% ist, wobei xxx ein von latex bzw. pdflatex lesbares Graphikformat
% ist, so kann man wie folgt ein Logo einf�gen:

% \pgfdeclareimage[height=0.5cm]{university-logo}{university-logo-filename}
% \logo{\pgfuseimage{university-logo}}



% Folgendes sollte gel�scht werden, wenn man nicht am Anfang jedes
% Unterabschnitts die Gliederung nochmal sehen m�chte.
\AtBeginSubsection[]
{
  \begin{frame}<beamer>
    \frametitle{Gliederung}
    \tableofcontents[currentsection,currentsubsection]
  \end{frame}
}


% Falls Aufz�hlungen immer schrittweise gezeigt werden sollen, kann
% folgendes Kommando benutzt werden:

%\beamerdefaultoverlayspecification{<+->}



\begin{document}

\begin{frame}
  \titlepage
\end{frame}

\begin{frame}
  \frametitle{Gliederung}
  \tableofcontents[pausesections]
  % Die Option [pausesections] k�nnte n�tzlich sein.
\end{frame}



% Da dies ein Vorlage f�r beliebige Vortr�ge ist, lassen sich kaum
% allgemeine Regeln zur Strukturierung angeben. Da die Vorlage f�r
% einen Vortrag zwischen 15 und 45 Minuten gedacht ist, f�hrt man aber
% mit folgenden Regeln oft gut.  

% - Es sollte genau zwei oder drei Abschnitte geben (neben der
%   Zusammenfassung). 
% - *H�chstens* drei Unterabschnitte pro Abschnitt.
% - Pro Rahmen sollte man zwischen 30s und 2min reden. Es sollte also
%   15 bis 30 Rahmen geben.



\section{Elliptische Kurven}

\subsection{Punktemenge}
\begin{frame}
\frametitle{Die Punktegrupppen-Menge.}
%\framesubtitle{Untertitel sind optional.}


\begin{definition}\textbf{Menge der Punktegruppe} \ $\left( \E (\K), + \right)$\\
Die Punktegruppe einer elliptischen Kurve $\E$ �ber $\K$ ist die Menge der Punkte, die auf der Kurve liegen und der Fernpunkt $\cO := (\infty,\infty)$.

$$ \E(\K) := \left\{ P=(x,y) \in \K^2 : y^2=x^3+ax^2+bx+c  \right\} \cup \left\{ \cO \right\} $$
\end{definition}

\end{frame}


\subsection{Beispiel}

%%%%%%%%%%%%%%%%%%%%%%%%%%%%%%%%%%%%%%%%%%%% Ohne Alles

\begin{frame}

\begin{example}
\begin{center}
$\E : y^2 = x^3-x$

\vspace{0.5cm}

\psset{unit=2.5cm} 
\begin{pspicture}(-1.3,-1.3)(1.3,1.3)
  \psset{linewidth=0.5pt}
  \psaxes[Dy=1,Dx=2]{->}(0,0)(-1.3,-1.3)(1.3,1.3)
  \psplot[linewidth=1pt]{-1}{0}{x x x mul mul x sub sqrt} 		% Linke Schlaufe
  \psplot[linewidth=1pt]{-1}{0}{x x x mul mul x sub sqrt -1 mul} 

  \psplot[linewidth=1pt]{1}{1.3}{x x x mul mul x sub sqrt} 					% Rechte Schlaufe
  \psplot[linewidth=1pt]{1}{1.3}{x x x mul mul x sub sqrt -1 mul} 

  \rput[lb](-1.08,0.28){$P$}
  \rput[lb](-0.21,0.48){$Q$}
%  \rput[lb](-0.65,0.64){$R$}
%  \rput[lb](1.24,0.71){$-(P+Q)$}
%  \rput[lb](1.26,-0.81){$P+Q$}
%  \rput[lb](1.03,0.03){$Q+R=-(Q+R)$}
%  \rput[lb](-0.02,0.13){$-(P+Q+R)$}

  \rput[lb](-1.2,-0.2){$-1$}
  \rput[lb](0.90,-0.2){$1$}

%  \psplot[linewidth=1pt,linecolor=red]{-1.3}{1.3}{x .25 mul .5 add} 							% Linie PQ
%  \psplot[linestyle=dashed]{-1.3}{1.3}{x -0.370060187 mul 0.370060187 add} 		% Linie -(Q+R)R
%  \psplot{-1.3}{1.3}{x -0.751812641 mul 0.120586693 add} 		% Linie -(P+Q+R)
%  \psplot[linestyle=dashed]{-1.3}{1.3}{x -0.131748658 mul 0.131748658 add} 		% Linie -(P+Q+R)


\end{pspicture}

\end{center}
\end{example}

\end{frame}


\begin{frame}

\begin{example}
\begin{center}
$\E : y^2 = x^3-x$

\vspace{0.5cm}

\psset{unit=2.5cm} 
\begin{pspicture}(-1.3,-1.3)(1.3,1.3)
  \psset{linewidth=0.5pt}
  \psaxes[Dy=1,Dx=2]{->}(0,0)(-1.3,-1.3)(1.3,1.3)
  \psplot[linewidth=1pt]{-1}{0}{x x x mul mul x sub sqrt} 		% Linke Schlaufe
  \psplot[linewidth=1pt]{-1}{0}{x x x mul mul x sub sqrt -1 mul} 

  \psplot[linewidth=1pt]{1}{1.3}{x x x mul mul x sub sqrt} 					% Rechte Schlaufe
  \psplot[linewidth=1pt]{1}{1.3}{x x x mul mul x sub sqrt -1 mul} 

  \rput[lb](-1.08,0.28){$P$}
  \rput[lb](-0.21,0.48){$Q$}
%  \rput[lb](-0.65,0.64){$R$}
  \rput[lb](1.24,0.71){$-(P+Q)$}
  \rput[lb](1.26,-0.81){$P+Q$}
%  \rput[lb](1.03,0.03){$Q+R=-(Q+R)$}
%  \rput[lb](-0.02,0.13){$-(P+Q+R)$}

  \rput[lb](-1.2,-0.2){$-1$}
  \rput[lb](0.90,-0.2){$1$}

  \psplot[linewidth=1pt,linecolor=red]{-1.3}{1.3}{x .25 mul .5 add} 							% Linie PQ
%  \psplot[linestyle=dashed]{-1.3}{1.3}{x -0.370060187 mul 0.370060187 add} 		% Linie -(Q+R)R
%  \psplot{-1.3}{1.3}{x -0.751812641 mul 0.120586693 add} 		% Linie -(P+Q+R)
%  \psplot[linestyle=dashed]{-1.3}{1.3}{x -0.131748658 mul 0.131748658 add} 		% Linie -(P+Q+R)


\end{pspicture}

\end{center}
\end{example}

\end{frame}

%%%%%%%%%%%%%%%%%%%%%%%%%%%%%%%%%%%%%%%%%%%% Mit Gerade....


\section{Axiome}
\subsection{Gruppenaxiome}

\begin{frame}
\frametitle{Axiome}
Ist $(\E(\K),+)$ eine \alert{kommutative} Gruppe?

  \begin{itemize}

	\item<2-> \textit{Abgeschlossenheit} \ $+: \E (\K)^2 \rightarrow \E (\K)$

  \item<6> \textit{Assoziativit�t} \ $P + (Q + R) = (P + Q) + R$

	\item<3-> \textit{Neutrales Element} \ $P + 0 = P$

	\item<4-> \textit{Inverse Elemente} \ $ -P$

	\item<5-> \textit{Kommutativit�t} \ $P + Q = Q + P$
  
  \end{itemize}

\end{frame}


\subsection*{Assoziativit�t}

% BILD 1

\begin{frame}
\begin{center}

\psset{unit=2.5cm} 
\begin{pspicture}(-1.3,-1.3)(1.3,1.3)
  \psset{linewidth=0.5pt}
  \psaxes[Dy=1,Dx=2]{->}(0,0)(-1.3,-1.3)(1.3,1.3)
  \psplot[linewidth=1pt]{-1}{0}{x x x mul mul x sub sqrt} 		% Linke Schlaufe
  \psplot[linewidth=1pt]{-1}{0}{x x x mul mul x sub sqrt -1 mul} 

  \psplot[linewidth=1pt]{1}{1.3}{x x x mul mul x sub sqrt} 					% Rechte Schlaufe
  \psplot[linewidth=1pt]{1}{1.3}{x x x mul mul x sub sqrt -1 mul} 

  \rput[lb](-1.08,0.28){$P$}
  \rput[lb](-0.21,0.48){$Q$}
  \rput[lb](-0.65,0.64){$R$}
  \rput[lb](1.24,0.71){$-(P+Q)$}
%  \rput[lb](1.26,-0.81){$P+Q$}
%  \rput[lb](1.03,0.03){$Q+R=-(Q+R)$}
%  \rput[lb](-0.02,0.13){$-(P+Q+R)$}

  \rput[lb](-1.2,-0.2){$-1$}
  \rput[lb](0.90,-0.2){$1$}

  \psplot[linewidth=1pt,linecolor=red]{-1.3}{1.3}{x .25 mul .5 add} 							% Linie PQ
%  \psplot[linestyle=dashed]{-1.3}{1.3}{x -0.370060187 mul 0.370060187 add} 		% Linie -(Q+R)R
%  \psplot{-1.3}{1.3}{x -0.751812641 mul 0.120586693 add} 		% Linie -(P+Q+R)
%  \psplot[linestyle=dashed]{-1.3}{1.3}{x -0.131748658 mul 0.131748658 add} 		% Linie -(P+Q+R)


\end{pspicture}
\end{center}
\end{frame}

% BILD 2

\begin{frame}
\begin{center}

\psset{unit=2.5cm} 
\begin{pspicture}(-1.3,-1.3)(1.3,1.3)
  \psset{linewidth=0.5pt}
  \psaxes[Dy=1,Dx=2]{->}(0,0)(-1.3,-1.3)(1.3,1.3)
  \psplot[linewidth=1pt]{-1}{0}{x x x mul mul x sub sqrt} 		% Linke Schlaufe
  \psplot[linewidth=1pt]{-1}{0}{x x x mul mul x sub sqrt -1 mul} 

  \psplot[linewidth=1pt]{1}{1.3}{x x x mul mul x sub sqrt} 					% Rechte Schlaufe
  \psplot[linewidth=1pt]{1}{1.3}{x x x mul mul x sub sqrt -1 mul} 

  \rput[lb](-1.08,0.28){$P$}
  \rput[lb](-0.21,0.48){$Q$}
  \rput[lb](-0.65,0.64){$R$}
  \rput[lb](1.24,0.71){$-(P+Q)$}
  \rput[lb](1.26,-0.81){$P+Q$}
  \rput[lb](1.03,0.03){$Q+R=-(Q+R)$}
%  \rput[lb](-0.02,0.13){$-(P+Q+R)$}

  \rput[lb](-1.2,-0.2){$-1$}
  \rput[lb](0.90,-0.2){$1$}

  \psplot[linewidth=1pt,linecolor=red]{-1.3}{1.3}{x .25 mul .5 add} 							% Linie PQ
  \psplot[linewidth=1pt,linecolor=green]{-1.3}{1.3}{x -0.370060187 mul 0.370060187 add} 		% Linie -(Q+R)
%  \psplot{-1.3}{1.3}{x -0.751812641 mul 0.120586693 add} 		% Linie -(P+Q+R)
%  \psplot[linestyle=dashed]{-1.3}{1.3}{x -0.131748658 mul 0.131748658 add} 		% Linie -(P+Q+R)


\end{pspicture}
\end{center}
\end{frame}

% BILD 3

\begin{frame}
\begin{center}
\psset{unit=2.5cm} 
\begin{pspicture}(-1.3,-1.3)(1.3,1.3)
  \psset{linewidth=0.5pt}
  \psaxes[Dy=1,Dx=2]{->}(0,0)(-1.3,-1.3)(1.3,1.3)
  \psplot[linewidth=1pt]{-1}{0}{x x x mul mul x sub sqrt} 		% Linke Schlaufe
  \psplot[linewidth=1pt]{-1}{0}{x x x mul mul x sub sqrt -1 mul} 

  \psplot[linewidth=1pt]{1}{1.3}{x x x mul mul x sub sqrt} 					% Rechte Schlaufe
  \psplot[linewidth=1pt]{1}{1.3}{x x x mul mul x sub sqrt -1 mul} 

  \rput[lb](-1.08,0.28){$P$}
  \rput[lb](-0.21,0.48){$Q$}
  \rput[lb](-0.65,0.64){$R$}
  \rput[lb](1.24,0.71){$-(P+Q)$}
  \rput[lb](1.26,-0.81){$P+Q$}
  \rput[lb](1.03,0.03){$Q+R=-(Q+R)$}

  \rput[lb](-1.2,-0.2){$-1$}
  \rput[lb](0.90,-0.2){$1$}

  \psplot[linewidth=1pt,linecolor=red]{-1.3}{1.3}{x .25 mul .5 add} 							% Linie PQ
  \psplot[linewidth=1pt,linecolor=green]{-1.3}{1.3}{x -0.370060187 mul 0.370060187 add} 		% Linie -(Q+R)
  \psplot[linewidth=1pt,linecolor=magenta]{-1.3}{1.3}{x -0.751812641 mul 0.120586693 add} 		% Linie -(P+Q+R)
%  \psplot[linewidth=1pt,linecolor=magenta]{-1.3}{1.3}{x -0.131748658 mul 0.131748658 add} 		% Linie -(P+Q+R)

  \rput[lb](-0.02,0.13){$-(P+Q+R)$}

\end{pspicture}
\end{center}
\end{frame}

% BILD 4

\begin{frame}
\begin{center}
\psset{unit=2.5cm} 
\begin{pspicture}(-1.3,-1.3)(1.3,1.3)
  \psset{linewidth=0.5pt}
  \psaxes[Dy=1,Dx=2]{->}(0,0)(-1.3,-1.3)(1.3,1.3)
  \psplot[linewidth=1pt]{-1}{0}{x x x mul mul x sub sqrt} 		% Linke Schlaufe
  \psplot[linewidth=1pt]{-1}{0}{x x x mul mul x sub sqrt -1 mul} 

  \psplot[linewidth=1pt]{1}{1.3}{x x x mul mul x sub sqrt} 					% Rechte Schlaufe
  \psplot[linewidth=1pt]{1}{1.3}{x x x mul mul x sub sqrt -1 mul} 

  \rput[lb](-1.08,0.28){$P$}
  \rput[lb](-0.21,0.48){$Q$}
  \rput[lb](-0.65,0.64){$R$}
  \rput[lb](1.24,0.71){$-(P+Q)$}
  \rput[lb](1.26,-0.81){$P+Q$}
  \rput[lb](1.03,0.03){$Q+R=-(Q+R)$}

  \rput[lb](-1.2,-0.2){$-1$}
  \rput[lb](0.90,-0.2){$1$}

  \psplot[linewidth=1pt,linecolor=red]{-1.3}{1.3}{x .25 mul .5 add} 							% Linie PQ
  \psplot[linewidth=1pt,linecolor=green]{-1.3}{1.3}{x -0.370060187 mul 0.370060187 add} 		% Linie -(Q+R)
  \psplot[linewidth=1pt,linecolor=magenta]{-1.3}{1.3}{x -0.751812641 mul 0.120586693 add} 		% Linie -(P+Q+R)
  \psplot[linewidth=1pt,linecolor=magenta]{-1.3}{1.3}{x -0.131748658 mul 0.131748658 add} 		% Linie -(P+Q+R)

  \rput[lb](-0.02,0.13){$-(P+Q+R)$}

\end{pspicture}
\end{center}
\end{frame}



\section{Additionsformeln}
\begin{frame}
\frametitle{Additionsformeln}
Formeln f�r \alert{Charakteristik}$(\K)$ $> 3$\\

$$ \E : y^2 = x^3 + ax + b $$
\end{frame}
\subsection{Inverse}
\begin{frame}
Die \textbf{Inversen} Elemente findet man mit einer \alert{Spiegelung} an der $X$-Achse:\\

$$ -P = -(x_P,y_P) = (x_P,-y_P) $$
\end{frame}
\subsection{Ungleiche Punkte}
\begin{frame}
Addieren \textbf{verschiedener} Punkte:\\

Seien $P = (x_P,y_P), Q = (x_Q,y_Q), R = (x_R,y_R)$, so da�\\

$$ P+Q = R $$
\end{frame}
\begin{frame}
Ist $x_P \neq x_Q$ k�nnen wir die Steigung der Gerade $\overline{PQ}$ berechnen:
$$ s = \frac{y_Q-y_P}{x_Q-x_P} $$

$R$ berechnet sich dann aus dieser Steigung $s$:

$$ x_R = s^2 - x_P - x_Q $$
$$ y_R = s \cdot (x_P - x_R) - y_P $$
\end{frame}
\subsection{Gleiche Punkte}
\begin{frame}
Addieren von 2 \textbf{gleichen} Punkte, also eine Punkt-\alert{Verdoppelung}.\\

$$ 2\cdot P = R $$
\end{frame}

\begin{frame}
Steigung der Tangente an $\E$ in $P$:
$$
s = \frac{3x_P^2+a}{2y_P}
$$
F�r $y_P \neq 0$ kann man aus der Steigung \textit{wie zuvor} $R$ berechnen:
$$ x_R = s^2 - 2x_P $$
$$ y_R = s \cdot (x_P - x_R) - y_P $$
\end{frame}

\section*{Spezielle Formeln}
\subsection*{Charakteristik 2}

\begin{frame}
\frametitle{Spezielle Formeln}
Formeln f�r \alert{Charakteristik}$(\K)$ $=2$\\

$$ \E : y^2 + yx = x^3 + ax^2 + b $$
\end{frame}

\begin{frame}

Durchs l�sen eines \alert{Gleichungsystems} ergibt sich:\\

  \begin{itemize}

	\item<1> \textit{Inverse} \ $-P = (x_P,y_P+x_P)$

  \item<2-> \textit{"`Steigung"'} \ $s := \frac{y_Q+y_P}{x_Q+x_P} $

	\item<3> \textit{Addieren} 
	$$ x_R = s^2 + s + x_P + x_Q + a $$
	$$ y_R = s \cdot (x_P + x_R) +x_R + y_P $$

	\item<4> \textit{Verdoppeln} 
	$$ x_R = s^2 + s + a $$
	$$ y_R = x_P^2 + s \cdot x_R +x_R $$

  \end{itemize}

\end{frame}


\section*{The End}
\begin{frame}
\frametitle{The End}
Vielen Dank f�r Ihre Aufmerksamkeit.
\end{frame}


\end{document}


