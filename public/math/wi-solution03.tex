\documentclass[11pt,a4paper,leqno]{article}

\usepackage{pst-plot, pstricks} 
\usepackage{fancybox,amssymb,color} 

\usepackage{amsfonts,german}
\usepackage{a4wide}
\usepackage{amssymb}
\usepackage{amsmath}
\usepackage{uebung}
\usepackage{texdraw}
%\usepackage{epsf}


\psset{unit=0.5cm} % Units f�r Graphen

\Nummer{3} % Blattnummer eingeben

%% Gruppe Haus Test Zusatz

\Anfang{1}{1}{1}{1} % Zaehler fuer  Gruppenuebung und Hausuebung eingeben

\Datum{08. Oktober 2004} % Datum eingeben

\parindent0mm



\begin{document}

\Kopf

\Uebungloes

%%%%%%%%%%%%%%%%%%%% Gruppenuebungen %%%%%%%%%%%%%%%%%%%%%%%%%%

\begin{Gruppenuebungen}


\Aufgabe % 1
\textbf{(Rechnen mit Vektoren im $\mathbb{R}^3$)}

Der Funktionsgraph ist streng monoton steigend, doch wird er im letzten Abschnitt ``flacher'', das hei�t die Steigung nimmt ab. $f'$ kann als die Neuverschuldung interpretiert werden.

\Aufgabe % 2
\textbf{(Physikunterricht)}

$f'$ beschreibt die Geschwindigkeit und $f''$ die Beschleunigung.

\Aufgabe % 3
\textbf{(Grenzwert der Sekantensteigung)}
\begin{eqnarray*}
f'(x_0) & = & \lim_{h \rightarrow 0} \ \frac{f(x_0+h) - f(x_0)}{h} \\
 & = & \lim_{h \rightarrow 0} \ \frac{5(x_0+h)^2 - 5x_0^2}{h} \\
 & = & \lim_{h \rightarrow 0} \ \frac{5x_0^2 + 10x_0h + 5h^2 - 5x_0^2}{h} \\
 & = & \lim_{h \rightarrow 0} \ 10x_0 + 5h \\
 & = & 10x_0 \\
 & = & 50. 
\end{eqnarray*}

\Aufgabe % 4
\textbf{(Wo ist $f$ differenzierbar?)}

\begin{itemize}
\item[(a)]
Es gibt bei der Funktion $f$ auf dem "Ubungsblatt $4$ Problemstellen. Der linke Rand. Die zwei ``Knicke'' vor dem (globalen) Maximum und die Sprungstelle nach dem Maximum. $f$ ist an keiner dieser Stellen differenzierbar und an der letzteren nicht mal stetig.\\
\item[(b)]
Ja, Differenzierbarkeit ist eine st"arkere Eigenschaft als Stetigkeit. Ist also $f$ in einem Punkt $x$ differenzierbar, so ist sie in diesem Punkt erst recht stetig. Umgekehrt kann $f$ an einer Unstetigkeitsstelle nie Differenzierbar sein.
\end{itemize}

\Aufgabe % 5
\textbf{(Ableiten)}

\begin{itemize}
\item[(a)]
$f'(x) = 15x^4 + 2$
\item[(b)]
$f'(x) = {3 \over 5}(x^3+2x)^{-\frac{2}{5}}(3x^2+2)$
\item[(c)]
$g'(t)= e^{t}(\sin{t} + \cos{t})$
\item[(d)]
$f'(z)=\frac{z(8z^7-\sin z)-2(z^8+\cos z)}{3z^3}$
\item[(e)]
$x'(t)=\cos({t^2})2t$
\item[(f)]
$g'(y)=0$
\item[(g)]
$g'(y) = 1$   
\end{itemize}


\Aufgabe % 6
\textbf{(Diese Aufgabe erschien uns nicht sinnvoll)}

\pagebreak

\Aufgabe % 7
\textbf{(Skizziere die Ableitung)}

Achtung: Diese Zeichnung ist nur sehr ungef"ahr!
\[ \begin{pspicture}(-3,-2)(6,3)
  \psaxes[labels=none]{->}(0,0)(-3,-2)(6,3)
  \psplot{-1}{0.5}{-1 x mul -1 add} 
  \psplot{0.5}{2}{x -2 add} 
  \psplot{2}{4}{0.05}
  \psplot{4}{5}{x -4.5 add x -4.5 add mul -10 mul 2.5 add}
  \psplot{5}{5.5}{-1 x mul 5 add}
\end{pspicture} \]

\Aufgabe % 8
\textbf{(Nullstelle, Extrema, Wendestellen)}
\begin{itemize}
\item[(a)]
$f(x)=-x^3+4x^2-3x=(x-0)(x-1)(x-3)$\\
Also sind die Nullstellen $0, 1, 3$.
\item[(b)]
$0 = f'(x) = -3x^2+8x-3= x^2-\frac{8}{3}x+1$\\
$f''(x_{min}) \neq 0 \neq f''(x_{max})$\\
Lokale Extrema sind bei $x_{min} = \frac{4}{3} - \frac{\sqrt 7 }{3}$ (Minimum) \ $x_{max} = \frac{4}{3} + \frac{\sqrt 7 }{3}$ (Maximum)
\item[(c)]
$0 = f''(x) = -6x+8$\\
$f'''(x_w) \neq 0$\\
Wendepunkte: $x_w = \frac{4}{3}$
\end{itemize}

\Aufgabe % 9
\textbf{(Lokale und globale Extrema)}

Wir untersuchen $f:[-4,4] \rightarrow \mathbb{R}: x \mapsto -\frac{1}{3}x^3-x^2+3x$.

\[ \begin{pspicture}(-4,-10)(10,1)
  \psaxes{->}(0,0)(-4,-8)(4,0)
  \psplot{-4}{3.1}{-1 3 div x x x mul mul mul -1 x x mul mul add 3 x mul add} 
\end{pspicture} \]
Also finden wir das lokale Minimum bei $x_{min} = -3$ und das globale Minimum bei $x_{Gmin} = 4$. Ein lokales und auch globales Maximum ist bei $x_{Gmax}=1$. 

\Aufgabe % 10 
\textbf{(Integrieren)}
\begin{itemize}
\item[(a)]
$\int \cos x \ dx = \sin x + c$
\item[(b)]
$\int x^2 +3x +4 \ dx = \frac{1}{3} x^3 + \frac{3}{2} x^2 + 4x + c$
\item[(c)]
$\int 5 \ dx = 5x + c$
\item[(d)]
$\int x\cdot e^x \ dx = (x-1)e^x  + c$
\item[(e)]
$\int \frac{1}{5}x \ dy = \frac{1}{5}xy + c$
\end{itemize}

\Aufgabe % 11 
\textbf{(Integrieren gleich Fl"acheninhalt?)}

Das gegebene Integral ergibt Null, da die oberen und unteren ``Fl"acheninhalte'' (die ja noch ein Vorzeichen tragen) sich aufheben.\\ Um den echten Fl"acheninhalt zu erhalten berechnen wir $\int^{2\pi}_{0} |\sin t| \ dt = 4$.

\[ \begin{pspicture}(-6,0)(6,3)
  \psaxes[labels=none]{->}(0,0)(-6,0)(6,3)
  \psplot{-6}{6}{180 3.14 div x mul sin} 
\end{pspicture} \]

\end{Gruppenuebungen}


\end{document}

%%% Local Variables:
%%% mode: latex
%%% TeX-master: t
%%% End: