\documentclass[11pt,a4paper,leqno]{article}
\usepackage{amsfonts,german}
\usepackage{a4wide}
\usepackage{amssymb}
\usepackage{amsmath}
\usepackage{uebung}
\usepackage{texdraw}
%\usepackage{epsf}

\newcommand{\smfrac}[2]%
{{\raise0.5ex\hbox{$\scriptstyle #1$} \kern-0.1em/\kern-0.15em \lower0.25ex\hbox{$\scriptstyle #2$}}}



\Nummer{1} % Blattnummer eingeben

%% Gruppe Haus Test Zusatz

\Anfang{1}{1}{1}{1} % Zaehler fuer  Gruppenuebung und Hausuebung eingeben

\Datum{08. Oktober 2004} % Datum eingeben

\parindent0mm



\begin{document}

\Kopf

\Uebungloes



%%%%%%%%%%%%%%%%%%%% Gruppenuebungen %%%%%%%%%%%%%%%%%%%%%%%%%%

\begin{Gruppenuebungen}


\Aufgabe % 1
\textbf{(Rechnen mit Vektoren im $\mathbb{R}^3$)}
\begin{itemize}

\item[(a)]
\[
\vec m _{Jan}  = \left( {\begin{array}{*{20}c}
   {140}  \\
   {700}  \\
   {35}  \\
\end{array}} \right) \hspace{1.5cm}
\vec m _{Feb}  = \left( {\begin{array}{*{20}c}
   {134}  \\
   {670}  \\
   {46}  \\
\end{array}} \right) \hspace{1.5cm}
\vec m _{Mar}  = \left( {\begin{array}{*{20}c}
   {137}  \\
   {685}  \\
   {40{\raise0.5ex\hbox{$\scriptstyle 1$}
\kern-0.1em/\kern-0.15em
\lower0.25ex\hbox{$\scriptstyle 2$}}}  \\
\end{array}} \right)
\]

\item[(b)]
$$
| \vec m_{Jan} | = 714.72 \hspace{1.5cm}
| \vec m_{Feb} | = 684.815 \hspace{1.5cm}
| \vec m_{Mar} | = 699.739
$$

\item[(c)]
\[
\det \left( {\begin{array}{*{20}c}
   {140} & {700} & {35}  \\
   {134} & {670} & {46}  \\
   {137} & {685} & {40{\raise0.5ex\hbox{$\scriptstyle 1$}
\kern-0.1em/\kern-0.15em
\lower0.25ex\hbox{$\scriptstyle 2$}}}  \\
\end{array}} \right) = 0
\]

\item[(d)]
Da die Determinante der Matrix aus c) $0$ ergibt hat das von den $3$ Vektoren aufgespannte Gebilde kein Volumen. Die Drei sind also linear abh"angig und spannen nur eine Fl"ache oder eine Gerade auf.
\end{itemize}

\Aufgabe % 2
\textbf{(Lineare Gleichungssysteme)}
\begin{itemize}
\item[(a)]
\[
\left( {\begin{array}{*{20}c}
   1 & 4 & 0 & 0 & {11}  \\
   0 & 0 & 3 & 2 & 1  \\
   2 & 0 & 0 & 0 & 0  \\
\end{array}} \right) \cdot \left( {\begin{array}{*{20}c}
   1  \\
   1  \\
   0  \\
   2  \\
   2  \\
\end{array}} \right) = \left( {\begin{array}{*{20}c}
   {27}  \\
   6  \\
   2  \\
\end{array}} \right)
\]

\item[(b)]
$$
x=40 \hspace{1cm} y=30 \hspace{1cm} z=20 \hspace{1cm} w=10 
$$

\end{itemize}

\Aufgabe % 3
\textbf{(W"urfelvolumen)}\\

Die Matrix der Vektoren, die einen W"urfel mit Kantenl"ange 2 im Raum aufspannen kann folgende Form haben:

\[
A = \left( {\begin{array}{*{20}c}
   2 & 0 & 0  \\
   0 & 2 & 0  \\
   0 & 0 & 2  \\
\end{array}} \right) \hspace{2cm}
\det A = 8
\]

\pagebreak

\Aufgabe % 4
\textbf{(Rechnen in $\mathbb{C}$)}
\begin{itemize}
\item[(a)]
$$
a+b=2+16i \hspace{1cm} a+d=5+18i \hspace{1cm} b+c=-1+3i \hspace{1cm} a+d+e=-3+18i
$$

\item[(b)]
$$
b-a=-8+2i \hspace{1cm} a-c=3+13i \hspace{1cm} e-b-c=-7-3i
$$

\item[(c)]
$$
a\cdot b= -78+24i \hspace{1cm} c\cdot d=66+22i \hspace{1cm} b\cdot e=24-72i \hspace{1cm} a\cdot c \cdot e = (52-16i)\cdot (-8) = -416+128i
$$

\item[(d)]
$$
\frac{b}{a}=\frac{1}{a\cdot \bar a}(b\cdot \bar a) = \frac{1}{37} (24+33i) \hspace{1cm} \frac{d}{c}=\frac{1}{20}(-33+11i)
$$

\item[(e)]
$$
|a|= 8.602 \hspace{1cm} |b|= 9.487 \hspace{1cm} |c|= 6.325
$$

\item[(f)]
$$
\bar a = 5-7i \hspace{0.5cm} \bar c = 2+6i \hspace{0.5cm} \bar d = -11i \hspace{0.5cm} \bar e = -8 \hspace{0.5cm} \overline{a+c} = 7-i \hspace{0.5cm} \overline{b\cdot d} = -99+33i \hspace{0.5cm} \overline{c-b}=5+15i
$$

\end{itemize}

\Aufgabe % 5
\textbf{(Kartesische- und Polardarstellung)}

\begin{itemize}
\item[(a)]
Eine komplexe Zahl $z$ mit der Polardarstellung $(r,\varphi)$ hat die folgende kartesische Darstellung $z = r\cdot (\cos \varphi + i \sin \varphi)$ oder $(r\cdot \cos \varphi,r\cdot \sin \varphi)$.
\item[(b)]
   \begin{itemize}
   \item[(i)]
$1+i \triangleright (\sqrt 2, \smfrac{\pi}{4})$
   \item[(ii)]
$-1+i \triangleright (\sqrt 2, \smfrac{3 \pi}{4})$
   \item[(iii)]
$i \triangleright (1, \smfrac{\pi}{2})$
   \item[(iv)]
$(\sqrt 2, \smfrac{\pi}{4}) \triangleright 1+i$
   \item[(v)]
$(3, \pi) \triangleright -3$
   \item[(vi)]
$(1, \smfrac{\pi}{3}) \triangleright \smfrac{1}{2}+\smfrac{\sqrt 3}{2} \ i$

   \end{itemize}
\end{itemize}


\Aufgabe % 6
\textbf{(Multiplikation mit $i$)}\\

Die Multiplikation mit $i$ entspricht geometrisch in der komplexen Zahlenebene einer Drehung um $\smfrac{\pi}{2}$ nach links.


\Aufgabe % 7
\textbf{(Komplexe Determinante)}\\
\[
\det \left( {\begin{array}{*{20}c}
   i & 0 & i  \\
   2 & i & {1 + i}  \\
   0 & {2 + 2i} & 1  \\
\end{array}} \right) =  - 1 + 0 + ( - 4 + 4i) - 0 - 0 - ( - 1 + i)(2 + 2i) =  - 1 + 4i
\]


\Aufgabe % 8
\textbf{(Unterschiede zwischen $\mathbb{R}^2$ und $\mathbb{C}$)}\\

Auf den komplexen Zahlen haben wir eine Multiplikation definiert, die es so zwischen Vektoren der reelen Ebene nicht gibt.

\end{Gruppenuebungen}


\end{document}
%%% Local Variables:
%%% mode: latex
%%% TeX-master: t
%%% End:
